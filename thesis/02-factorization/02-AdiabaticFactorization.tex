\section{Adiabatic Factorization Algorithm}

The problem of integer factorization can be formulated as a
constrained search over pairs of natural numbers $p$ and $q$
such that
\begin{equation}
	N = p \times q.
	\label{eq:integer_factorization}
\end{equation}
In the context of adiabatic quantum computation we aim to
encode this constraint into the ground state of a problem Hamiltonian,
allowing the solution to emerge through adiabatic evolution.

To encode candidate solutions $p$ and $q$, we adopt a binary representation
of natural numbers. Any natural number $N$ can be expressed as:
\begin{equation}
	N = \sum_{j=0}^{n_\text{bits} - 1} 2^j x_j,
	\label{eq:binary_integer}
\end{equation}
where each $x_j \in \{0,1\}$ and the bit string $\mathbf{x} = x_{n_\text{bits}-1} \dots x_0$
represents the binary encoding of $N$. In our approach, we exploit the fact that the factors
$p$ and $q$ of an odd composite number can be rewritten as:
\begin{equation}
	\begin{cases}
		p = 2p' + 1 \\
		q = 2q' + 1
	\end{cases} .
	\label{eq:factors_simplification}
\end{equation}
It can be proved that the $n_p$ and $n_q$ are an upper bound on the number of qubits
required to represent $p'$ and $q'$, respectively:
\begin{equation}
	\begin{cases}
		n_p = m\,\big(\lfloor \sqrt{N} \rfloor_o\big) - 1 \\[2ex]
		n_q = m\bigg(\left\lfloor \dfrac{N}{3} \right\rfloor \bigg) - 1
	\end{cases} ,
	\label{eq:factors_num_bits}
\end{equation}
where $\lfloor a \rfloor \big(\lfloor a \rfloor_o\big)$ denotes the largest (odd)
integer not larger than a, while $m(b)$ denotes the smallest number of bits
required for representing $b$~\cite{peng_quantum_2008}. Then, the adiabatic factorization
algorithm will make use of $n = n_p + n_q$ qubits. The full quantum state 
\begin{equation}
	\ket{\Psi} = \ket*{\Psi_{p'}} \otimes \ket*{\Psi_{q'}},
	\label{eq:full_quantum_state}
\end{equation}
where $\ket*{\Psi_{p'}} = \ket*{\psi_{1}}\otimes \cdots \otimes \ket*{\psi_{n_p}}$
and $\ket*{\Psi_{q'}} = \ket*{\psi_{n_p+1}}\otimes \cdots \otimes \ket*{\psi_{n_p +n_q}}$.

To encode the factorization constraint into the adiabatic quantum framework, we follow
the approach of Ref.~\cite{hegade_digitized_2021} and define the objective function:
\begin{equation}
	f(p,q) = \big(N - p \times q \big)^2,
\end{equation}
such that its global minimum $f(p,q)=0$ corresponds to valid factors. Translating this
into a Hamiltonian acting on the computational basis, we obtain the quadratic problem Hamiltonian:
\begin{equation}
	\hat{H}_\mathrm{QP} = \bigg[ N \1 - \bigg( \sum_{\ell=1}^{n_p} 2^\ell \hat{x}_\ell + \1 \bigg)
	\bigg( \sum_{m=1}^{n_q} 2^m \hat{y}_m + \1 \bigg) \bigg]^2\,,
	\label{eq:quadratic_problem_hamiltonian}
\end{equation}
where $\hat{x}_\ell = \dfrac{\1 - \hat{\sigma}_\ell^z}{2}$ and $\hat{y}_m = \dfrac{\1 - \hat{\sigma}_m^z}{2}$
are the number operators acting on the qubits encoding $p'$ and $q'$, respectively. The solution
to the factorization problem is encoded in the ground state of $\hat{H}_\mathrm{QP}$.

The initial state of the system is prepared as:
\begin{equation}
	\ket{\psi(0)} = \ket{+}^{\otimes n}\,,
	\label{eq:initial_state}
\end{equation}
where $\ket{+} = \frac{1}{\sqrt{2}} (\ket{0} + \ket{1})$ is the eigenstate of $\hat{\sigma}^x$,
corresponding to the ground state of the initial Hamiltonian $\hat{H}_0$ defined in Eq.~\ref{eq:transverse_field_hamiltonian}.
Following the adiabatic theorem, if the evolution from $\hat{H}_0$ to $\hat{H}_\mathrm{QP}$
is slow enough, the system will remain in the instantaneous ground state, reaching the ground state of
$\hat{H}_\mathrm{QP}$ at the end of the protocol. This final state encodes the solution to the factorization
problem.

Despite its coneptual clarity, the Hamiltonian $\hat{H}_\mathrm{QP}$ includes high-order
multi-qubit interactions, such as three- and four-body terms of the form
$\hat{\sigma}_\ell^z \hat{\sigma}_m^z \hat{\sigma}_k^z$ and $\hat{\sigma}_\ell^z \hat{\sigma}_m^z \hat{\sigma}_k^z \hat{\sigma}_n^z$.
These many-body interactions are difficult to implement, since they require to bring
all involved qubits together and make them interact in a controled way, resulting in
prone-to-error processes.

To mitigate this issue, we propose a linearized problem Hamiltonian
inspired by the same factorization condition, defined as:
\begin{equation}
	\hat{H}_\mathrm{LP} = N \1 - \bigg( \sum_{\ell=1}^{n_p} 2^\ell \hat{x}_\ell + \1 \bigg)
	\bigg( \sum_{m=1}^{n_q} 2^m \hat{y}_m + \1 \bigg)\,.
	\label{eq:linear_problem_hamiltonian}
\end{equation}
$\hat{H}_\mathrm{LP}$ contains only two-body interaction terms, which can be efficiently
simulated. However, unlike $\hat{H}_\mathrm{QP}$, the desired solution is not the ground state of $\hat{H}_\mathrm{LP}$,
but an eigenstate with eigenvalue zero. Since the adiabatic theorem applies to any non-degenerate
eigenstate, not necessarily the ground state, we hypothesize that this eigenstate can still be
targeted by an appropriate adiabatic or variational algorithm.

\subsection{Digitized Adiabatic Quantum Factorization}
Although the adiabatic model is often formulated in terms of continuous time evolution, it
can be simulated efficiently using gate-based quantum computation {\color{red} (ADD REFERENCES)}
through a process known as digitization.

In the digitized approach, the continuous adiabatic evolution governed by a time-dependent
Hamiltonian as of Eq.~\ref{eq:adiabatic_passage} is approximated by a sequence of quantum
gates through trotterization, breaking the total evolution into small time slices. Each slice
is implemented as a layer in a quantum circuit, simulating the adiabatic trajectory step
by step.

This method was successfully demonstrated in Ref.~\cite{hegade_digitized_2021}, where
the authors implemented a digitized adiabatic factorization algorithm on superconducting
hardware.

\subsection{QAOA Applied to Factorization}
The Quantum Approximate Optimization Algorithm can be viewed as a shortcut to adiabaticity.
Rather than performing a slow, continuous evolution, QAOA uses a fixed-depth quantum
circuit composed of alternating unitaries derived from the mixing and problem Hamiltonians.
The parameters of these unitaries are optimized variationally to prepare a state that
approximates the solution.

As shown in Ref.~\cite{diez-valle_universal_2025}, QAOA and quantum annealing share universal
structural properties, and there is compelling evidence that smooth annealing paths can be constructed
from QAOA optimal parameters. This supports the interpretation of QAOA as a digitized
and variationally optimized version of an adiabatic process.

In this approach, our proposal is to use the linearized Hamiltonian from equation
Eq.~\ref{eq:linear_problem_hamiltonian} in the layers of the QAOA algorithm, and a cost
function that encodes the factorization solution in its minimum, such as
$\langle \hat{H}_\mathrm{QP} \rangle$ or any other Hamiltonian that accomplish this.

