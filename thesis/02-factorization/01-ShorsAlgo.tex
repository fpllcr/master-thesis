\section{Shor's Factorization Algorithm}

In computer science, an algorithm is considered to be efficient when the number of steps
of the algorithm grows as a polynomial in the input size. For the problem of integer factorization,
the input is a semiprime number $N$, and the input size is measured as the number of bits
required to represent it, i.e., $\log N$. The best known classical algorithm for factorizing
scales superpolynomially with input size, making the problem computationally hard for large $N$.
\begin{equation}
    \mathcal{O} \bigg( \exp \big( c(\log N)^{1/3} (\log \log N)^{2/3} \big) \bigg).
    \label{eq:field_sieve_scaling}
\end{equation}

In 1994, Peter Shor introduced a quantum algorithm that factorizes integers in polynomial time,
marking a landmark result in quantum computing. Shor's algorithm runs in time
\begin{equation}
    \mathcal{O} \big( (\log N)^3 \big),
    \label{eq:shor_scaling}
\end{equation}
dramatically outperforming the best classical method~\cite{nielsen00}. The algorithm relies on the quantum
circuit model, making it a gate-based approach to factorization and one of the strongest motivations
for the development of quantum computers.

The core idea of Shor's algorithm is to reduce factorization to the problem of order finding:
given a number $a$ coprime to $N$, find the smallest integer $r$ such that
\begin{equation}
    a^r = 1 \mod N.
    \label{eq:order_finding}
\end{equation}
Once the order $r$ is known, under certain conditions, one can recover a nontrivial factor of
$N$ using elementary number-theoretic arguments~\cite{nielsen00}.

The quantum part of the algorithm is used to efficiently find this order $r$ using
period-finding techniques. This is achieved by preparing a quantum superpositioin and applying the
quantum Fourier transform (QFT) to extract information about the period of the function
$f(x) = a^x \mod N$. The classical post-processing step then uses continued fractions to extract $r$
and attempt to derive the prime factors of $N$.

Despite its theoretical significance, implementing Shor's algorithm at scale requires a
large number of qubits, long coherence times, and fault-tolerant error correction,
which remain beyond the reach of current quantum hardware. As such, while Shor's algorithm remains
the most efficient known method for factorizing in the long-term quantum regime, alternative approaches
--- such as adiabatic and variational algorithms --- are being explored for use in the NISQ era.