\section{Shor's Factorization Algorithm}

{\color{blue} ALAN: First we have to briefly introduce the Shor's algorithm. We do not need a
very detailed explanation, but it would be nice to provide a didactic discussion about the algorithm
and its relevance. To this end, we can use the book by Nielsen and Chuang. In this section,
we make clear that it is a gate-based way to do factorization.}

In computer science, an algorithm is considered to be efficient when the number of steps
of the algorithm grows as a polynomial in the input size. For the problem of integer factorization,
the input is a semiprime number $N$, and the input size is measured as the number of bits
required to represent it, i.e., $\log N$. The best known classical algorithm for factoring
scales superpolynomially with input size, making the problem computationally hard for large $N$.
\begin{equation}
    \exp \bigg[ \bigg( (64/9)^{1/3} + o(1) \bigg) (\log N)^{1/3} (\log \log N)^{2/3} \bigg].
    \label{field_sieve_scaling}
\end{equation}

In 1994, Peter Shor introduced a quantum algorithm that factors integers in polynomial time,
marking a landmark result in quantum computing. Shor's algorithm runs in time
\begin{equation}
    \mathcal{O} \big( (\log N)^3 \big),
    \label{shor_scaling}
\end{equation}
dramatically outperforming the best classical method. The algorithm relies on the quantum
circuit model, making it a gate-based approach to factorization and one of the strongest motivations
for the development of quantum computers.