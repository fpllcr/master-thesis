\chapter{Factorization Algorithms}
\label{Chapter:Factorization}

\dropcap{T}he integer factorization problem consists of
decomposing a given semiprime number $N$ into its prime constituents. While seemingly simple,
no known classical algorithm can factor large integers efficiently, and the best classical
methods --such as the general number field sieve-- scale superpolynomially with input
size~\cite{montgomery_cwi_1994}. This computational asymmetry forms the foundation of widely
used cryptographic protocols, most notably RSA, which relies on the practical difficulty of
factoring large semiprimes to ensure security.

The emergence of quantum algorithms has dramatically shifted the landscape of computational
complexity associated with factorization. In particular, Shor's algorithm demonstrated that
quantum computers can solve the problem in polynomial time, meaning a direct threat to
RSA-based cryptography. However, implementing Shor's algorithm requires fault-tolerant quantum,
hardware which remains out of reach in the current NISQ era.


This chapter explores two quantum approaches to the factorization problem. First, we review
Shor's algorithm and its quantum Fourier transform-based structure. Then, we introduce
alternative strategies based on adiabatic quantum computation, which may offer more viable paths
towards factorization on near-term quantum devices.