\section{Adiabatic Quantum Computation}
\label{Section:AQC}
The adiabatic theorem of quantum mechanics provides the foundation for an alternative model
of quantum computation based on continuous-time evolution. Consider a time-dependent Hamiltonian
$H(t)$ with a discrete and non-degenerate spectrum. Thus we can define its instantaneous
eigenstates and eigenenergies by
\begin{equation}
    \hat{H}(t) \ket{n(t)} = E_n (t) \ket{n(t)}.
    \label{eq:instantaneous_eigenstates}
\end{equation}

\noindent The adiabatic theorem states that a quantum system initially prepared in an eigenstate
$\ket{n(0)}$ of  $H(0)$, will remain in its instantaneous eigenstate $\ket{n(t)}$ throughout
the evolution, provided that the spectrum remains gapped and the change is sufficiently
slow~\cite{sarandy_consistency_2004,albash_adiabatic_2018}. This last condition is generally formulated as follows:
\begin{equation}
    \max_{0 \leq t \leq T} \bigg| \dfrac{\bra{k} \dot{\hat{H}} \ket{n}}{g_{nk}} \bigg| \ll \min_{0 \leq t \leq T} \big| g_{nk} \big|,
    \label{eq:adiabatic_condition}
\end{equation}
where $T$ is the total evolution time and $g_{nk}$ represents the energy gap between levels $n$ and $k$: 
\begin{equation}
    g_{nk}(t) \equiv E_n(t) - E_k(t)
    \label{eq:energy_gap}
\end{equation}

\begin{figure}[h]
    \centering
    \includegraphics[width=0.75\textwidth]{01-introduction/figs/adiabatic_theorem.pdf}
    \caption{Schematic of an adiabatic passage. The orange line represents the Hamiltonian's ground state, while
    the green a Hamiltonian's eigenstate different from the ground state.}
    \label{fig:adiabatic_passage}
\end{figure}

Despite the adiabatic theorem works for any eigenstate, in practice it is customary to focus on
ground state adiabatic passages. This is because ground states are typically more robust against
decoherence and thermal excitations, as they are energetically isolated from higher-energy levels.
Moreover, for many computational problems, particularly those related to optimization problems,
it is possible to construct a Hamiltonian $\hat{H}_\mathrm{P}$ whose ground state encodes
the solution to the instance under consideration.

In the adiabatic quantum computation model, the Hamiltonian is interpolated between an
initial Hamiltonian $\hat{H}_0$, with a known and easily preparable ground state, and the problem
Hamiltonian $\hat{H}_\mathrm{P}$, according to a schedule~\cite{albash_adiabatic_2018}:
\begin{equation}
    \hat{H}(t) = \big[1 - s(t)\big] \hat{H}_0 + s(t) \hat{H}_\mathrm{P}, \quad s(t) \in [0,1],
    \label{eq:adiabatic_passage}
\end{equation}
where $s(t)$ is a smooth, monotonic function such that $s(0)=0$ and $s(T)=1$. The problem
Hamiltonian $\hat{H}_\mathrm{P}$ is designed so that its ground state corresponds to the solution
of the computational task.

Aharonov et al. proved that this model is computationally equivalent to the standard circuit
model~\cite{aharonov_adiabatic_2004}, meaning that any adiabatic process
can, in principle, be reproduced digitally with comparable efficiency. This equivalence opens
the door to a variety of gate-based approaches that draw inspiration from adiabatic evolution
while remaining fully compatible with the circuit model of quantum computation.

\subsection{Adiabatic Quantum Annealers}
An important and widely studied application of adiabatic evolution in quantum
devices is Adiabatic Quantum Annealing (AQA), where the goal is to find low-energy
configurations of a classical cost function mapped onto a quantum Hamiltonian. AQA is most
often applied to Quadratic Unconstrained Binary Optimization (QUBO) problems~\cite{kadowaki_quantum_1998},
which can be exactly reformulated as a 2-body interaction Ising Hamiltonian of the form
\begin{equation}
    H_\mathrm{P} = \sum_i h_i \sigma_i^z + \sum_{i<j} J_{ij} \sigma_i^z \sigma_j^z\,,
    \label{eq:ising_hamiltonian}
\end{equation}
where $h_i$ are local fields, $J_{ij}$ represent coupling strengths between qubits, and
$\sigma_i^z$ are Pauli-Z operators acting on the $i$-th qubit. The task is to drive the system
from an initial, easily preparable ground state towards the ground state of $H_\mathrm{P}$,
which encodes the optimal solution to the problem at hand.

In a typical AQA protocol, the evolution begins with an initial transverse-field Hamiltonian
\begin{equation}
    H_0 = \sum_i \sigma_i^x\,.
    \label{eq:transverse_field_hamiltonian}
\end{equation}
whose ground state is straightforward to prepare. The system is then evolved according to the interpolation
scheme in equation (\ref{eq:adiabatic_passage}). The aim is to adiabatically steer the system from the easily
preparable ground state $H_0$ to the ground state of $H_\mathrm{P}$, which encodes the solution to the
problem of interest.

This procedure is straightforward for problems that can be expressed in the QUBO form, as they can be mapped to
an Ising Hamiltonian with only two-body interactions. However, the factorization problem presented in
chapter~\ref{Chapter:Factorization} does not directly fall into the QUBO class, as its Hamiltonian contains
three- and four-body interaction terms. As will be discussed later, this limitation can be addressed by adopting
an alternative formulation of the problem Hamiltonian that avoids these high-order terms while preserving the encoded solution.