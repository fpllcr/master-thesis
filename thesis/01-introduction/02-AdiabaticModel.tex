\section{Adiabatic Quantum Computation}
\label{Section:AQC}

The adiabatic theorem of quantum mechanics provides the foundation for an alternative model
of quantum computation based on continuous-time evolution. Consider a time-dependent Hamiltonian
$H(t)$ with a discrete and non-degenerate spectrum. Thus we can define its instantaneous
eigenstates and eigenenergies by $$H(t) \ket{n(t)} = E_n (t) \ket{n(t)}.$$

\noindent The adiabatic theorem states that a quantum system initially prepared in an eigenstate
$\ket{n(0)}$ of  $H(t)$, will remain in its instantaneous eigenstate $\ket{n(t)}$ throughout
the evolution, provided that the spectrum remains gapped and the change is sufficiently
slow~\cite{sarandy_consistency_2004,albash_adiabatic_2018}. This last condition is generally formulated as follows:
$$\max_{0 \leq t \leq T} \bigg| \dfrac{\bra{k} \dot{H} \ket{n}}{g_{nk}} \bigg| \ll \min_{0 \leq t \leq T} \big| g_{nk} \big|,$$
where $T$ is the total evolution time and $g_{nk}$ represents the energy gap between levels $n$ and $k$: 
$$g_{nk}(t) \equiv E_n(t) - E_k(t)$$

Even though, it is customary in quantum computation to focus on
ground state adiabatic passages. This is because ground states are typically more robust against
decoherence and thermal excitations, as they are energetically isolated from higher-energy levels.
Moreover, for many computational problems, particularly those related to optimization problems,
it is possible to construct a problem Hamiltonian $H_\mathrm{P}$ whose ground state encodes
the solution to the instance under consideration.

In the adiabatic quantum computation model, the Hamiltonian is interpolated between an
initial Hamiltonian $H_0$, with a known and easily preparable ground state, and the problem
Hamiltonian $H_\mathrm{P}$, according to a schedule~\cite{albash_adiabatic_2018}:
$$H(t) = (1 - s(t)) H_0 + s(t) H_P, \quad s(t) \in [0,1],$$
where $s(t)$ is a smooth, monotonic function such that $s(0)=0$ and $s(T)=1$. The problem
Hamiltonian $H_\mathrm{P}$ is designed so that its ground state corresponds to the solution
of the computational task.

(Maybe talk about AQC can be efficiently simulated in the circuit model).


Talk here about adiabatic theorem and evolutions. Then, introduce the concept/definition of
adiabatic quantum computation, where we define the problem Hamiltonians and etc. To this end,
we can get results by Sarandy [https://doi.org/10.1007/s11128-004-7712-7] and references therein,
and some discussion done in [Rev. Mod. Phys. 90, 015002 (2018)] to make our life easier.

\lipsum[1-2]

\subsection{Adiabatic Quantum Annealers}

Here we discuss about the adiabatic quantum aneealers, where the problem Hamiltonian is the Ising Hamiltonian (which is the focus of our applications). To this end, maybe we can use the review paper to get some key discussions [Rev. Mod. Phys. 90, 015002 (2018)].

$CZ_{2}(\theta_{n})$

\lipsum[1-2]





