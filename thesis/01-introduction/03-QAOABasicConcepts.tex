\section{QAOA}
\label{Section:QAOA}

The Quantum Approximate Optimization Algorithm (QAOA) is a variational quantum algorithm
designed to solve discrete optimization problems by approximating the ground state of a cost Hamiltonian.
Originally proposed by Farhi et al. in 2014~\cite{farhi_quantum_2014}, QAOA is well-suited to the capabilities
and limitations of near-term quantum devices, as it operates with relatively shallow circuits
and allows for classical optimization of quantum parameters.

The main idea of QAOA is to construct a quantum ansatz state by alternating the application of
two types of operators derived from two Hamiltonians:
\begin{itemize}
    \item A cost Hamiltonian $H_C$, which encodes the optimization problem (often
    in the form of an Ising model).
    \item A mixing Hamiltonian $H_M$, which typically promotes exploration of the solution
    space and is often chosen as:
    \begin{equation}
        H_M = \sum_i \sigma_i^x
        \label{eq:mixing_hamiltonian}
    \end{equation}
\end{itemize}

{\color{red} FELIP: Add some picture of the QAOA structure}

Given an initial state $\ket{\psi_0} = \ket{+}^{\otimes n}$, which is the ground state of
$H_M$, the QAOA ansatz after $p$ layers is defined as:
\begin{equation}
    \ket{\psi(\bm{\gamma, \beta})} = e^{-i \beta_p H_M} e^{- i \gamma_p H_C} \cdots
    e^{-i \beta_1 H_M} e^{- i \gamma_1 H_C} \ket{+}^{\otimes n}
\end{equation}
where $\bm{\gamma} = (\gamma_1, \dots, \gamma_p)$ and $\bm{\beta} = (\beta_1, \dots, \beta_p)$
are real variational parameters. We then determine the expectation value of $H_C$ as:
\begin{equation}
    F_p (\bm{\gamma}, \bm{\beta}) = \bra{\psi_p (\bm{\gamma}, \bm{\beta})} H_C \ket{\psi_p (\bm{\gamma}, \bm{\beta})},
    \label{eq:cost_function}
\end{equation}
which is precisely the cost function to be minimized by varying the QAOA parameters.

{\color{red} FELIP: Talk here about why QAOA is believed to be a promising algorithm. Also,
Juanjo's work (Universal Resouces for QAOA and Quantum Annealing) contains interesting
statements like "... universal properties common to QAOA and QA... there is evidence
of smooth annealing paths constructed from QAOA optimal parameters..."}