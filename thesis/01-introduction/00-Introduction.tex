\chapter{Introduction}
\label{introduction}

\dropcap{Q}uantum computing has emerged as the most promising paradigm for solving
problems that are classically intractable. One of the most celebrated achievements
is the discovery of the Shor's algorithm, which demonstrated that the integer factorization
problem can be solved exponentially faster on a quantum computer than with the best-known
classical algorithm. However, the full realization of this algorithm is currently out of reach
due to quantum hardware limitations, known as the Noisy Intermediate-Scale Quantum (NISQ) era.

The constraints imposed by the NISQ era have motivated alternative methods to attempt to
solve the same problems by using fewer quantum resources. One such approach is the Adiabatic
Quantum Computation (AQC), which takes advantage of the robustness of adiabatic evolution to
bring the system from a known state to a final state that encodes the solution of a problem. Another
approach is the Variational Quantum Algorithm (VQA), which is a hybrid algorithm that uses quantum
hardware for state evolution, and classical routines for optimization purposes.

In this work, we propose and analyze a version of the Digitized Adiabatic Quantum Factorization (DAQF)
algorithm that incorporates a variation in the cost Hamiltonian. By this, we aim to provide a more
resource-efficient and scalable method for integer factorization.

\let\clearpage



