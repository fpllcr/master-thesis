\subsection{Initial State}
\label{Section:InitialState}

In line with the adiabatic theorem, which states that a quantum system prepared in an eigenstate of a slowly varying Hamiltonian will remain in its instantaneous eigenstate, we aim to initialize the system in an eigenstate of the mixing Hamiltonian $\hat{H}_\mathrm{M}$. Ideally, this initial state should occupy a similar region of the spectrum as the target solution under the problem Hamiltonian $\hat{H}_\mathrm{P}$, so that the variational evolution requires minimal spectral “rearrangement.” The specific choice of initial state depends on the protocol used:

\begin{itemize}
    \item \textbf{\texttt{standard} protocol} ($\hat{H}_\mathrm{P} = \hat{H}_\mathrm{QP}$): The solution corresponds to the ground state of the problem Hamiltonian. To reflect this structure, we initialize the system in the ground state of the mixing Hamiltonian, $$\ket{\psi (0)} = \ket{+}^{\otimes n},$$ which is both easy to prepare and spectrally aligned with low-energy states of $\hat{H}_\mathrm{P}$.

    \item \textbf{\texttt{linear} protocols} ($\hat{H}_\mathrm{P} = \hat{H}_\mathrm{LP}$): Here, the correct solution lies near the center of the spectrum rather than at its bottom. Then, we choose an eigenstate of $\hat{H}_\mathrm{M}$ that is approximately centered in its spectrum, $$\ket{\psi (0)} = \ket{+-+-+\cdots}.$$ For problems with an odd number of qubits, this configuration cannot be perfectly centered, but QAOA's non-adiabatic nature tolerates this slight asymmetry without degrading performance.
\end{itemize}
