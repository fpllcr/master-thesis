\chapter{Methodology}
\label{Chapter:Methodology}

\dropcap{T}he Quantum Approximate Optimization Algorithm has been extensively investigated
since its introduction, leading to a broad landscape of theoretical analyses, practical 
implementations, and performance-enhancement techniques. Numerous variants have been proposed,
ranging from modified ansatze to parameter initialization heuristics, and adaptive layer-scaling
procedures. This diversity reflects both the flexibility of the QAOA framework and the
complexity of identifying implementations that perform well accross different problem
instances.

Given this wide range of possible approaches, it is essential to precisely specify
the algorithmic configuration adopted in this work. In the following sections, we describe
the chosen setup, selected to ensure that the comparison between our proposed protocol and
the standard QAOA implementation is both meaningful and fair. The configuration is based
on the best-performing practices for the standard protocol, so that any observed differences
in performance can be attributed to the change in the problem Hamiltonian rather than to
auxiliary implementation details.

Although several software frameworks for quantum circuit simulation exist, such as PennyLane
and Qiskit, this work employs a direct state-vector simulation using explicit matrix
multiplications implemented in Numpy. This approach avoids the overhead of gate-by-gate
simulation, providing a more efficient and transparent means of evaluations QAOA circuits
at the scale considered. For completeness and reproducibility, the full implementation
is available in a public GitHub repository\footnote{\url{https://github.com/fpllcr/master-thesis}},
where all details of the numerical procedures and data analysis can be examined.