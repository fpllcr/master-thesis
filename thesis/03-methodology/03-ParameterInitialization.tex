\section{Parameter initialization}
\label{Section:ParameterInitialization}

As discussed in Sec.~\ref{Section:IncrementalApproach}, adding a new QAOA layer requires a suitable heuristic to initialize the additional variational parameters. Beyond random initialization, two specific strategies have been considered in this work:

\begin{itemize}
    \item \textbf{Parameter interpolation.}  
    This method, adopted by Zhou et al.~\cite{zhou_quantum_2020}, linearly interpolates the trajectory traced by the optimized parameters at depth $p$ to generate an initial guess for depth $p+1$. Such interpolation is particularly effective for problems like MaxCut and QUBO, where the optimal parameters often evolve monotonically in an adiabatic-like fashion.

    \item \textbf{Alternative heuristic.}  
    This approach, which achieved superior performance for both the standard and linearized protocols in our experiments, initializes the new parameters as
    \begin{equation}
        \begin{cases}
            \gamma_{p+1} \leftarrow \widetilde{\gamma}_{p}, \\
            \beta_{p+1} \leftarrow 0.
        \end{cases}
        \label{eq:initialization_heuristic}
    \end{equation}
\end{itemize}

Two factors motivated the adoption of this alternative strategy.  
First, the interpolation heuristic performs poorly for the linearized protocol, frequently leading to trapping in local minima.  
Second, integer factorization differs fundamentally from MaxCut and QUBO problems: there is no evidence that optimal parameters follow adiabatic-like trajectories, and our results confirm that they do not (see Fig.~\ref{fig:optimizer_comparison} and Fig.~\ref{fig:optimizer_parameter_comparison}).  
The fact that the alternative heuristic also outperforms interpolation for the standard protocol ensures a fair basis for comparison, as both protocols are evaluated under the most favorable optimization setup for the reference (standard) implementation.
