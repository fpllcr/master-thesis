\chapter{Results}
\label{Chapter:Results}

\dropcap{T}his chapter presents the numerical experiments performed to compare the performance
of the standard QAOA protocol with the proposed linearized protocol. The standard protocol is
evaluated in its original formulation, while the linearized protocol is analyzed under two different
cost functions to determine which provides the most favorable performance. Table~\ref{tab:protocols_overview}
summarizes the three protocols considered throughout this chapter. 

\begin{table}[h]
    \centering
    \begin{tabular}{@{}cccc@{}}
        \toprule
        Protocol            & Mixing Hamiltonian    & Problem Hamiltonian   & Cost function \\
        \midrule
        \texttt{standard}            & $\hat{H}_\mathrm{M}$   & $\hat{H}_\mathrm{QP}$ & $\langle \hat{H}_\mathrm{QP} \rangle$ \\
        \texttt{linear\_quadratic}   & $\hat{H}_\mathrm{M}$   & $\hat{H}_\mathrm{LP}$ & $\langle \hat{H}_\mathrm{QP} \rangle$ \\
        \texttt{linear\_abs}         & $\hat{H}_\mathrm{M}$   & $\hat{H}_\mathrm{LP}$ & $\langle |\hat{H}_\mathrm{LP}| \rangle$ \\
        \bottomrule
    \end{tabular}
    \caption{Overview of QAOA protocols analyzed in this work.}
    \label{tab:protocols_overview}
\end{table}

The primary evaluation metric throughout this chapter is the \emph{fidelity}, which directly
corresponds to the probability of successfully obtaining the correct factors of the integer
to be factorized. Using fidelity as a baseline metric allows for a clear and consistent
comparison between protocols.

All practical aspects of the QAOA setup---including initialization heuristics and the choice of
classical optimizer---were fixed and justified in Chapter~\ref{Chapter:Methodology} to ensure
fair and consistent comparisons. The focus here is exclusively on evaluating and contrasting
the protocols across problem instances of increasing size under these controlled conditions.

As representative examples, we present detailed results for the biprime integers $N = 25$, $N = 77$,
and $N = 143$. A complete set of results can be found in the accompanying GitHub repository
and is summarized in the appendix~\ref{Chapter:Appendix}. Table~\ref{tab:instances_overview} provides an overview of
these selected test cases, including relevant problem parameters and solution encodings.
\begin{table}[h]
    \centering
    \begin{tabular}{@{}ccccccccc@{}}
        \toprule
        $N$ & $n$ (\# qubits) & $p$ ($p'$) & $q$ ($q'$) & $n_p$ & $n_q$ & $p_\mathrm{bitstring}$ & $q_\mathrm{bitstring}$ & solution(s) \\
        \midrule
        25  & 4          & 5 (2)      & 5 (2)      & 2     & 2     & 10                     & 10                     & 0101                  \\
        77  & 6          & 7 (3)      & 11 (5)     & 2     & 4     & 11                     & 0101                   & 111010                \\
        143 & 8          & 11 (5)     & 13 (6)     & 3     & 5     & 101                    & 00110                  & 10101100, 01110100    \\
        \bottomrule
    \end{tabular}
    \caption{Overview of biprime test instances used as representative examples.}
    \label{tab:instances_overview}
\end{table}

